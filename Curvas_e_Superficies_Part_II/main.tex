\documentclass[12pt]{article}
\usepackage{graphicx}
\usepackage[utf8]{inputenc}
\usepackage[english]{babel}
\usepackage{fullpage}
\usepackage{listings}
\usepackage{xcolor}
\usepackage{url}
\usepackage[linesnumbered,ruled,vlined]{algorithm2e}
\usepackage{enumitem}
\usepackage{mathrsfs}
\usepackage{amssymb}
\usepackage{amsmath}
\usepackage{enumitem}
\usepackage{hyperref}
\usepackage{float}
\usepackage{sbc-template}


\definecolor{mygreen}{rgb}{0,0.6,0}

\hypersetup{
    colorlinks=true,
    linkcolor=cyan,
    urlcolor=cyan}

\newcounter{problem}
\newcounter{solution}

\pagestyle{plain}
\thispagestyle{plain}

\newtheorem{prop}{Proposição}
\newtheorem{ex}{Exemplo}[section]
\newtheorem{theorem}{Teorema}
\newtheorem{corollary}{Corolário}[theorem]
\newtheorem{lemma}[theorem]{Lemma}
\newtheorem{definition}{Definição}

\lstset{ % lstlisting
    language=Python,
    frame=tb, % draw a frame at the top and bottom of the code block
    tabsize=4, % tab space width
    showstringspaces=false, % don't mark spaces in strings
    commentstyle=\color{mygreen}, % comment color
    keywordstyle=\color{blue}, % keyword color
    stringstyle=\color{red}, % string color
    numbers=left, 
    numbersep=9pt,
    backgroundcolor=\color{black!5}, % set backgroundcolor
    basicstyle=\footnotesize,% basic font setting
}

\newcommand{\furl}[1]{\footnote{\url{#1}}}

\newcommand\Problem{%
  \stepcounter{problem}%
  \textbf{Problema \theproblem:}
  \setcounter{solution}{0}%
}

\newcommand\TheSolution{%
  \textbf{Solução:}\\%
}

\newcommand\ASolution{%
  \stepcounter{solution}%
  \textbf{Solução \thesolution:}\\%
}
% \parindent0in
\parskip 1.5em

\title{Curso de Curvas e Superfícies - Parte II}

\author{Wellington José Leite da Silva\inst{1}}

\address{Escola de Matemática Aplicada da FGV (EMAP), Brazil}

\date{}

\begin{document}

\maketitle

%\section*{Sumário}

%\textbf{\nameref{s1}}

%\textbf{\nameref{s2}}
%\vspace{4.0mm}

%\textbf{\nameref{s3}}
%\vspace{4.0mm}


\section*{Apresentação}\label{s1}
Continuando com o que foi construído na parte I apresentamos aqui uma linha de aprendizado do curso de curvas e superfícies apresentando definições, teoremas, exemplos e etc. Separados em \todo{Welly}{Adicionar as seções}. 

Com intuído de auxiliar o aprendizado aos tópicos apresentados e fornecer uma forma de visualização computacional apresentamos exemplos com códigos em \textit{SageMath}~\cite{sagemath}. Aqui seguimos o livro \cite{bookmain} como principal e o \cite{manfredo} como complementar. Adicionando sempre que possível, exemplos de visualizações em \textit{SageMath}. As implementações, códigos usados para as mesmas assim como o \textit{Tex} deste documento se concentram no repositório curvas-superficies~\furl{https://github.com/wellington36/curvas-superficies} que está disponível abertamente no github.

Todos os códigos apresentados nos exemplos podem ser facilmente generalizados para outros casos, é recomendável como forma de aprendizado rodar os códigos apresentados com outros exemplos de escolha do leitor.

\section{Superficies}

\bibliographystyle{sbc}
\bibliography{referencias}

\end{document}
